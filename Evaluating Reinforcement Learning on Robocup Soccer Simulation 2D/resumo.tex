A liga de simulação 2D de futebol do RoboCup é uma das mais maduras da competição, iniciada em 1996. Uma das tarefas a ser realizada pelos agentes inteligentes é a linha de defesa. Ela consiste em saber a hora certa entre interceptar a bola ou fazer uma marcação inteligente (bloqueando prováveis chutes ou passes). Uma investida errada do agente pode levar a um drible ou passe do atacante e um provável gol. Aprendizagem supervisionada e algoritmos determinísticos são as técnicas mais usadas pelas 5 melhores equipes para resolver o problema. Pesquisas recentes usando Aprendizagem Profunda por Reforço (APR) para treinar agentes autônomos em sistemas multiagentes superaram os agentes com base em algoritmos supervisionados ou determinísticos. Um exemplo é a equipe CYRUS2019 que produziu jogadores defensivos treinados com o APR, alcançando em terceiro lugar na RoboCup 2019. Este trabalho compara três técnicas de APR em agentes defensivos com base no CYRUS2019, adaptando o \textit{Half Field Offensive} para ser um ambiente semelhante aos da OpenAI GYM e aplicar a melhor técnica aos agentes do time RoboCIn2d.